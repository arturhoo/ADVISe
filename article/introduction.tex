\section{Introduction} 

In recent decades there was a significant growth of biological data generated by experimental techniques such as the new generation DNA sequencing, protein sequencing and protein structure determination. Much of these data are organized and made publicly available to the scientific community in biological databases over the Internet. According to \cite{lesk2005database} these repositories not only store biological raw data but also relevant information related to them such as literature data, protein function, relationship between a protein and its encoding gene, among other metadata.

Given that these biological databases are growing at very high rates, most of these metadata are automatically assigned. In the majority of the cases, with no laboratory experiments at all, the roles of most genes in several organisms have been reported by homology propagation \cite{brenner1999errors}. To ensure that these annotations remain reliable, studies about the confiability of the entries as well as measures of confidence should be developed. Many studies have called the attention to errors rates in the biological databases annotations \cite{devos2001intrinsic,green2005genome,jones2007estimating,schnoes2009annotation,hung2010detect}

In fact, the automatic identification of theses errors is still an open problem and several challenges have to be faced. Without laboratory experiments to verify automatically assigned annotations, it is impossible to know for certain. However, most of the studies present comparisons of diverse functional annotation methods and show they are widely incompatible what place a rough upper bound on their accuracy. 

A major step toward automatic error detection is a description of how and to what extent biological databases entries annotations evolve. In other words, we have to be capable to understand why some entries seem to be more stable and and others more volatile and what are the factors that determines this different behaviours.

The research and development of models and algorithms as well as visualization ressources are very promissing toward understanding how biological databases evolve. Interactive visualizations can be specially powerful to represent in a macro/micro perspective this voluminous, high-dimensional and complex datasets and to help users to unveil trends and exceptions in those data sets. 

\subsection{Enzyme annotations}

By the late 1950's it had become evident that the nomenclature of enzymology, in a period when the number of known enzymes was increasing rapidly, was getting out of hand. In many cases the same enzymes became known by several different names, while conversely the same name was sometimes given to different enzymes. Many of the names conveyed little or no idea of the nature of the reactions catalysed, and similar names were sometimes given to enzymes of quite different types. To meet this situation, the General Assembly of the International Union of Biochemistry (IUB) decided, in consultation with the International Union of Pure and Applied Chemistry (IUPAC), to set up an International Commission on Enzymes. Its objective was to consider the classification and nomenclature of enzymes and coenzymes, their units of activity and standard methods of assay, together with the symbols used in the description of enzyme kinetics. The Commission prepared a report,in 1961 and it was adopted and has been widely used in scientific journals, textbooks, etc. since then. The size of the Enzyme Commission number (EC number) list has increased steadily since the publication of the first report and also many corrections were done.

The  EC number is a numerical classification scheme for enzymes, based on the chemical reactions they catalyze. Every enzyme code consists of four numbers separated by periods. Those numbers represent a hierarchical progressively finer classification of the catalized reaction. For example, the code: 3.4.21.4 is a:
\begin{description}
\item [3:] hydrolase, which means the enzyme breaks a chemical bond using a water molecule.
\item [3.4:] peptidase, which means the broken bond is a peptide bond, i.e., a bond between amino acids in a protein chain.
\item [3.4.21]: endopeptidase, because it breaks an intra-chain peptide bond.
\item [3.4.21.4:] trypsin, because enzyme has the specificity of cutting close the residues arginine and lisine.
\end{description}

When a new enzyme is annotated, one can add from one to four levels of the EC number, depending on the detail of existing knowledge. In the better case, we know all about the catalyzed reaction as well as the specific substrates and products involved. However, in many cases all we know is that the molecule is an enzyme. In this case, the annotation is left "-.-.-.-". An EC number "3.4.21.-", for instance, means we don't know enzyme substrates specifically although we have information about the reaction catalyzed.
