\documentclass{vgtc}                          % final (conference style)
%\documentclass[review]{vgtc}                 % review
%\documentclass[widereview]{vgtc}             % wide-spaced review
%\documentclass[preprint]{vgtc}               % preprint
%\documentclass[electronic]{vgtc}             % electronic version

\usepackage{mathptmx}
\usepackage{graphicx}
\usepackage{times}

\onlineid{0}

\vgtccategory{Research}

\vgtcinsertpkg

\title{Visualizing the dynamics of enzyme annotations in Uniprot/SwissProt}

\author{Sabrina A. Silveira\thanks{e-mail: sabrina@dcc.ufmg.br}\\ %
        \scriptsize Federal University of Minas Gerais %
\and Artur Rodrigues\thanks{e-mail: artur@dcc.ufmg.br} \\ %
        \scriptsize Federal University of Minas Gerais\\ %
\and Raquel C. de Melo-Minardi\thanks{e-mail:raquelcm@dcc.ufmg.br}\\ %
     \scriptsize Federal University of Minas Gerais
\and \\Wagner Meira Jr.\thanks{e-mail:meira@dcc.ufmg.br}\\ %
     \scriptsize Federal University of Minas Gerais
}

\teaser{
\\
  \includegraphics[width=4cm]{sample.eps}
  \caption{Teaser figure}
}

\abstract{
Write abstract
}

\CCScatlist{
Information visualization, Bioinformatics, Database dynamics, Enzymes, EC number, UniProt, SwissProt, Annotation, Processing.\\
}

\begin{document}

\firstsection{Introduction}

\maketitle

\section{Introduction} 

\subsection{Biological databases}

In recent decades there was a significant growth of biological data generated by experimental techniques such as the new DNA sequencing technologies, protein sequencing and protein structure determination. Much of these data are organized and made available to the scientific community in biological databases over the Internet. According to \cite{lesk2005database} these repositories not only store biological raw data but also relevant information related to them such as literature data, protein function, relationship between a protein and its encoding gene, among others.

\cite{brenner1999errors}
\cite{devos2001intrinsic}
\cite{green2005genome}
\cite{jones2007estimating}
\cite{schnoes2009annotation}
\cite{hung2010detect}

\subsection{Enzyme anotations}

\subsection{The dynamics of annotations}

\section{Technique}


\begin{equation}
 \sum_{j=1}^{z} j = \frac{z(z+1)}{2}
\end{equation}


\begin{table}[!h]
  \caption{Vis Paper Acceptance Rate}
  \label{vis_accept}
  \scriptsize
  \begin{center}
    \begin{tabular}{cccc}
      Year & Submitted & Accepted & Accepted (\%)\\
    \hline
      1994 &  91 & 41 & 45.1\\
      1995 & 102 & 41 & 40.2\\
      1996 & 101 & 43 & 42.6\\
      1997 & 117 & 44 & 37.6\\
      1998 & 118 & 50 & 42.4\\
      1999 & 129 & 47 & 36.4\\
      2000 & 151 & 52 & 34.4\\
      2001 & 152 & 51 & 33.6\\
      2002 & 172 & 58 & 33.7\\
      2003 & 192 & 63 & 32.8\\
      2004 & 167 & 46 & 27.6\\
      2005 & 268 & 88 & 32.8\\
      2006 & 228 & 63 & 27.6
    \end{tabular}
  \end{center}
\end{table}

\begin{figure}[htb]
  \centering
  \includegraphics[width=1.5in]{sample.eps}
  \caption{Sample illustration.}
\end{figure}

\section{Discussions}

\subsection{TODO insights}

\begin{itemize}


\item Falar que o que não é mudança é o mais numeroso (constantes).

\item Falar que nas versões 11 a 15 foram identificadas mudanças drásticas bem numerosas (deleções de 4 níveis). Colocar quantas entradas sofreram tais mudanças. Colocar exemplos biológicos do que isso significa (de acordo com a resposta do UniProt).

\item Matriz triangular superior: entradas que sofreram mais generalizações que especializações.

\item Matriz triangular inferior: entradas que sofreram mais especializações que generalizações. Mostra que ao longo dos anos entradas vão ganhando anotações.

\end{itemize}

\section{Conclusion}

\acknowledgements{This work was supported by the Brazilian agencies Coordena\c{c}\~{a}o de Aperfei\c{c}oamento de Pessoal de N\'{i}vel Superior (CAPES), Conselho Nacional de Desenvolvimento Cient\'{i}fico e Tecnol\'{o}gico (CNPq), Funda\c{c}\~{a}o de Amparo \`{a} Pesquisa do Estado de Minas Gerais (FAPEMIG), Financiadora de Estudos e Projetos (FINEP) and Pr\'o-Reitoria de Pesquisa da Universidade Federal de Minas Gerais.}

\bibliographystyle{abbrv}
\bibliography{template}
\end{document}
